\documentclass[../thesis.tex]{subfiles}

\begin{document}

\chapter{Linking Predictive and Prescriptive Analytics for Healthcare Services}

\section{Introduction}
Should reference this paper: \ur{https://reader.elsevier.com/reader/sd/pii/S1877050920305159?token=99F091AB53C524561BFA24D591EFFE3EE11E50C17611B296B5505EC90CEF59EB6B509802B3727AD8463ABF43DB889594&originRegion=eu-west-1&originCreation=20230228155212}

\section{Linking Paradigms}\label{sec:linking}
One of the four research aims is to determine `how can linking predictive and prescriptive analytics increase the reliability of the models?'. This section aims to link the CART results with deterministic and two-stage stochastic models together to ensure the results are consistent. Two methods will be explored, firstly by using each end node and determining the number of patients of each specialty and hospital using the average LOS from each end node. The second method will take each end node and the specific LOS for each specialty and hospital within the node. These will then be summed together to form the D$_{s,r}$ parameter. These two methods will be run on both the regression and the classification trees, using the Microsoft Excel implementation. The results will be compared on a year to year basis, as well as a three year range. Similar to previous experiments, the models will be run with a maximum run time of 100 seconds and a tolerance of 0.01\%. The deterministic and two-stage stochastic models will be used to calculated the VSS. The demands utilised within this section and the resulting heatmaps can be seen within Appendix \ref{app:linkeddemands}.

\subsection{Regression Tree and Average LOS}
The first method utilises the regression tree generated in Section \ref{sec:regressiontrees}, Figure \ref{fig:finalregtree}. This tree generated 30 leaf nodes, and therefore 30 patient groupings. For each node, the demand can be calculated using the Average LOS determined by the node. Using Equation \eqref{eq:treedemand1}, the average demand can be calculated:

\begin{equation}\label{eq:treedemand1}
        D_{s,r} = \sum\limits_{h \in r} D_{s,h} = \frac{\text{Number of Patients}_{s,h}*\text{Node Average LOS}}{1096}
\end{equation}

Table \ref{tab:Node2Regression} displays the process of determining the average bed demand for node 2. Node 2 is the second left leaf node on Figure \ref{sec:regressiontrees}, where a patient falls under the following eleven criterion:
\begin{enumerate}
    \item Admission method $\neq$ Other - transferred from another hospital
    \item Admission method $\neq$ Elective - waiting list
    \item Admission method $\neq$ Elective - booked
    \item Specialty $\neq$ Accident \& Emergency
    \item Admission method $\neq$ Elective - planned
    \item Hospital $\neq$ Ysbyty Ystrad Fawr
    \item Age Group $\neq$ 65 - 69
    \item Age Group $\neq$ 70 - 74
    \item Specialty $\neq$ Trauma \& Orthopaedic
    \item Age Group $\neq$ 75 - 79
    \item Specialty = Care Of The Elderly
\end{enumerate}

For this end node, there are three hospitals included, all of which are the COTE specialty accounting for 8776 patients. The final column of Table \ref{tab:Node2Regression} refers to the average daily bed demand across three years worth of data. Each nodes demands are consolidated for each specialty and hospital and the used for the overall demand, can be seen within Table \ref{apptab:LinkedDemands1}.
\begin{table}[h!]
    \centering\scalebox{0.8}{
    \begin{tabular}{ccccc}\toprule 
    \textbf{Hospital} & \textbf{Specialty} & \textbf{Count}& \textbf{Average LOS} & \textbf{Average Daily Demand}\\\midrule
    Nevill Hall Hospital &  Care Of The Elderly &     2696 &11.393 & 28.0251168 \\
    Royal Gwent Hospital &  Care Of The Elderly &     6076 & 11.393& 63.1604635 \\
    Ysbyty Aneurin Bevan &  Care Of The Elderly &        4 & 11.393&  0.0415803\\ \bottomrule
    \end{tabular}}
    \caption{Regression Tree Node 2 - Average LOS}
    \label{tab:Node2Regression}
\end{table}


Table \ref{tab:Results1} presents the results from the demands which can be found within Table \ref{tab:LinkedDemands1} in the Appendix. The same four scenarios, as listed in Table \ref{tab:scenarios1}, were applied to the demand figures. The VSS can be calculated to be 2.80\% with a saving of $\pounds$28,922.24. In comparison to the Table \ref{tab:eiveesveevdettwostageresults1}, there is a difference in deterministic solution of approximately 1.51\%, with the regression tree deploying fewer numbers of beds and nurses. In the two-stage stochastic model, the number of beds deployed in the first stage remain the same, although the locations of the beds differ, the other decision variables differ. The differences in the objective function, bed and staff deployment could be due to the regression tree R$^{2}$ score of 34.28\% which is not accounting for longer LOS's. 

\begin{table}[h!]
    \centering
    \begin{tabular}{cccccl}\toprule
 & \multicolumn{2}{l}{\textbf{Total Beds}} & \multicolumn{2}{c}{\textbf{Total Staff}} & \multirow{2}{*}{\textbf{Objective Value ($\pounds$)}}\\ \cmidrule(lr){2-3} \cmidrule(lr){4-5}
 & xbed           & ubed          & xstaff         & ustaff         \\ \midrule
    Deterministic      & 1019 & - & 615 & - & 998,157.80 = EV \\ \midrule
    Stochastic &1003& 119& 591 & 93&  1,034,073.36 = RP \\ \midrule
    Test A & 1019 & 103 & 615 & 123 & 1,062,995.60 = EEV \\\bottomrule
    \end{tabular}
    \caption{Deterministic and Two-Stage Stochastic Results - Regression Tree and Average LOS - Three Years}
    \label{tab:Results1}
\end{table}

The year to year planning can also be calculated with the regression tree nodes to determine how the model performs on a yearly basis. Equation \eqref{eq:treedemand2} displays how each of the yearly demands are generated.
\begin{equation}\label{eq:treedemand2}
        D_{s,r,year} = \sum\limits_{h \in r} D_{s,h,year} = \frac{\text{Number of Patients}_{s,h}* {\text{Node Average LOS}}_{year}}{\text{Number of Days in Year}}
\end{equation}
Table \ref{tab:Results2} displays the EV, RP and EEV values for each year, with the demands given in Tables \ref{apptab:LinkedDemands2a} and \ref{apptab:LinkedDemands2b}. For the year 2019 - 2020, the total capacity had to be increased for the hospital Ysbyty Aneurin Bevan, meaning additional beds would be required from other age ranges in order to meet the demand. This would suggest the average LOS for the year 2019 - 2020 is too large for this hospital compared to previous years.

Comparing the results to Table \ref{tabeiveesveevdetstocresults2}, suggest the model using the average LOS for the nodes under predict the required demands. The deterministic difference ranges from 0.24\% to 6.52\%, showing the average LOS across all three years may not yield beneficial results. The locations of the bed placements can be seen within Appendix \ref{} \hl{reference}.



\begin{table}[h!]
    \centering
    \begin{tabular}{ccccccl}\toprule
 & \multirow{2}{*}{\textbf{Year}}& \multicolumn{2}{l}{\textbf{Total Beds}} & \multicolumn{2}{c}{\textbf{Total Staff}} & \multirow{2}{*}{\textbf{Objective Value ($\pounds$)}}\\ \cmidrule(lr){3-4} \cmidrule(lr){5-6}
&& xbed           & ubed          & xstaff         & ustaff         \\ \midrule
     \multirow{3}{*}{Deterministic} & 2017 - 2018 & 975  & - & 567  & - &  939,334.04 =  EV$_{17-18}$ \\ 
      & 2018 - 2019 &1003 & - & 582 & - & 975,890.84 =  EV$_{18-19}$ \\
      & 2019 - 2020 & 1008 & - & 594 & - &  993,691.28 =  EV$_{19-20}$\\ \midrule
     \multirow{3}{*}{Stochastic} & 2017 - 2018 & 961 & 120 & 540 & 81 & 973,915.42 =  RP$_{17-18}$ \\ 
      & 2018 - 2019 & 982 &  118 & 573 & 87 & 1,013,177.66 =  RP$_{18-19}$ \\
      & 2019 - 2020 &993 & 118 &570 &87 & 1031731.4 =  RP$_{19-20}$\\ \midrule    
     \multirow{3}{*}{Test A} & 2017 - 2018 & 975 & 101 &  567 & 120 & 1,000,582.82
=  EEV$_{17-18}$ \\ 
      & 2018 - 2019& 1003 & 92& 582 & 105 &1,029,686.36 =  EEV$_{18-19}$ \\
      & 2019 - 2020 & 1008 & 99 & 594 &120 & 1,059,168.92 =  EEV$_{19-20}$\\ \bottomrule       
    \end{tabular}
    \caption{Deterministic and Two-Stage Stochastic Results - Regression Tree and Average LOS - Year Specific}
    \label{tab:Results2}
\end{table}



The third and final experiment will take the specific LOS for each end node and specialty, hospital and year combination. This should mitigate the issues caused in the previous experiment where YAB did not have sufficient capacity within the UB$_{h}^{\textnormal{max, bed}}$ constraint. Equation \eqref{eq:treedemand3} displays the formulation of the demands, with the overall demands listed in Tables \ref{apptab:LinkedDemands3a} and \ref{apptab:LinkedDemands3b}.
\begin{equation}\label{eq:treedemand3}
        D_{s,r,year} = \sum\limits_{h \in r} D_{s,h,year} = \frac{\text{Number of Patients}_{s,h}*\text{Node Average LOS}_{s,h,year}}{\text{Number of Days in Year}}
\end{equation}

The results listed in Table \ref{tab:Results3} have objective functions much similar to those in the original experiment. The difference in the deterministic values range from 0.82 to 2.24\%. \hl{add conclusion here after next section to see what that result is...}. 

\begin{table}[h!]
    \centering
    \begin{tabular}{ccccccl}\toprule
 & \multirow{2}{*}{\textbf{Year}}& \multicolumn{2}{l}{\textbf{Total Beds}} & \multicolumn{2}{c}{\textbf{Total Staff}} & \multirow{2}{*}{\textbf{Objective Value ($\pounds$)}}\\ \cmidrule(lr){3-4} \cmidrule(lr){5-6}
&& xbed           & ubed          & xstaff         & ustaff         \\ \midrule
     \multirow{3}{*}{Deterministic} & 2017 - 2018 & 1027   & - &  582 & - & 991,514.84 =  EV$_{17-18}$ \\ 
      & 2018 - 2019 & 1004 & - &585 &  - &  985,028.20 =  EV$_{18-19}$ \\
      & 2019 - 2020 & 997 & - & 585 & - &   974,160.20 =  EV$_{19-20}$\\ \midrule
     \multirow{3}{*}{Stochastic} & 2017 - 2018 & 1001  & 129  & 561 & 87 & 1,025,039.30  =  RP$_{17-18}$ \\ 
      & 2018 - 2019 & 981 & 122  & 570 & 87 & 1,020,358.10 =  RP$_{18-19}$ \\
      & 2019 - 2020 & 979 & 117 & 561 & 84 & 1,011,335.52 =  RP$_{19-20}$\\ \midrule    
     \multirow{3}{*}{Test A} & 2017 - 2018 & 1027  & 97 &  582 & 105 & 1,047,382.34=  EEV$_{17-18}$ \\ 
      & 2018 - 2019& 1004 & 95 & 585 & 105 &1,040,391.70 =  EEV$_{18-19}$ \\
      & 2019 - 2020 & 997 & 95 & 585  & 120 & 1,036,865.48 =  EEV$_{19-20}$\\ \bottomrule       
    \end{tabular}
    \caption{Deterministic and Two-Stage Stochastic Results - Regression Tree and Average Year Specific LOS}
    \label{tab:Results3}
\end{table}




\subsection{Regression Tree and Specific LOS}
The second method also utilises the regression tree shown in Figure \ref{fig:finalregtree}. Instead of using average LOS for each of the 30 end nodes, the average LOS will be determined for each hospital and specialty within that node. Equation \eqref{eq:treedemand4} displays how each of the demands are generated.

\begin{equation}\label{eq:treedemand4}
        D_{s,r} = \sum\limits_{h \in r} D_{s,h} = \frac{\text{Number of Patients}_{s,h}*{\text{Specific LOS}_{s,h}}}{1096}
\end{equation}

Table \ref{tab:Node2Regressionb} presents the second node within the regression tree and determines how each of the demands are generated within this node. These results in comparison with Table \ref{tab:Node2Regression}, have shown that using specific hospital and specialty LOS increase the demand by one bed overall in RGH. The generated demands can be viewed in Table \ref{apptab:LinkedDemands4}.

\begin{table}[h!]
    \centering\scalebox{0.8}{
    \begin{tabular}{ccccc}\toprule 
    \textbf{Hospital} & \textbf{Specialty} & \textbf{Count}& \textbf{Average LOS} & \textbf{Average Daily Demand}\\\midrule
    Nevill Hall Hospital &  Care Of The Elderly &     2696 &11.412 & 28.0729927 \\
    Royal Gwent Hospital &  Care Of The Elderly &     6076 & 11.554& 64.0510949 \\
    Ysbyty Aneurin Bevan &  Care Of The Elderly &        4 & 6.250&  0.0228102\\ \bottomrule
    \end{tabular}}
    \caption{Regression Tree Node 2 - Specific LOS}
    \label{tab:Node2Regressionb}
\end{table}

Table \ref{tab:Results4} presents the results for the deterministic and two-stage stochastic model, with tests the EEV also being calculated to determine the VSS.

\begin{table}[h!]
    \centering
    \begin{tabular}{cccccl}\toprule
 & \multicolumn{2}{l}{\textbf{Total Beds}} & \multicolumn{2}{c}{\textbf{Total Staff}} & \multirow{2}{*}{\textbf{Objective Value ($\pounds$)}}\\ \cmidrule(lr){2-3} \cmidrule(lr){4-5}
 & xbed           & ubed          & xstaff         & ustaff         \\ \midrule
    Deterministic      & 1014 & - & 576 & - & 982,943.12 = EV \\ \midrule
    Stochastic &984& 130& 564 & 90&  1,014,511.04 = RP \\ \midrule
    Test A & 1014 & 93 & 576 & 105 & 1,035,641.54 = EEV \\\bottomrule
    \end{tabular}
    \caption{Deterministic and Two-Stage Stochastic Results - Regression Tree and Specific LOS - Three Years}
    \label{tab:Results4}
\end{table}

Comparing to the regression tree with average LOS results, the deterministic and two-stage stochastic objective value's were lower in the specific LOS model. This suggests that rather than using node averages, if the specific LOS is used, more cost savings can be generated. The VSS produced a saving of $\pounds$21,130.50 per day (2.08\%). 

These results can be analysed on a year to year basis with the specific LOS produced from each regression tree node (Tables \ref{apptab:LinkedDemands5a} and \ref{apptab:LinkedDemands5b}). To calculate the demands for each specialty and region for each year Equation \eqref{eq:treedemand5} can be applied.


\begin{equation}\label{eq:treedemand5}
        D_{s,r,year} = \sum\limits_{h \in r} D_{s,h,year} = \frac{\text{Number of Patients}_{s,h}*{\text{Specific LOS}_{s,h,year}}}{{\text{Number of Days in Year}}}
\end{equation}

Table \ref{tab:Results5} presents the results after OpenSolver tool has optimised the bed and staffing numbers based on the demand values. \hl{ADD something here?} 

\begin{table}[h!]
    \centering
    \begin{tabular}{ccccccl}\toprule
 & \multirow{2}{*}{\textbf{Year}}& \multicolumn{2}{l}{\textbf{Total Beds}} & \multicolumn{2}{c}{\textbf{Total Staff}} & \multirow{2}{*}{\textbf{Objective Value ($\pounds$)}}\\ \cmidrule(lr){3-4} \cmidrule(lr){5-6}
&& xbed           & ubed          & xstaff         & ustaff         \\ \midrule
     \multirow{3}{*}{Deterministic} & 2017 - 2018 & 1015 & - & 558 & - & 970,852.96=  EV$_{17-18}$ \\ 
      & 2018 - 2019 &987 & - & 546 & - & 953,385.52 =  EV$_{18-19}$ \\
      & 2019 - 2020 &  985 & - & 540 & -&  949,538.80 = EV$_{19-20}$\\\midrule
     \multirow{3}{*}{Stochastic} & 2017 - 2018 & 992 & 128&537 & 93 & 1,006,058.76  =  RP$_{17-18}$ \\ 
      & 2018 - 2019 &968&121& 531& 90& 986,649.90=  RP$_{18-19}$ \\
      & 2019 - 2020 & 963  &  129& 531& 93& 991,981.12=  RP$_{19-20}$\\ \midrule
      \multirow{3}{*}{Test A}& 2017 - 2018 & 1015 & 99 &558&102 & 1,027,247.56= EEV$_{17-18}$\\
      & 2018 - 2019 &994& 93&552&105&1,008,706.66= EEV$_{18-19}$\\
      & 2019 - 2020 & 987 & 99 & 540 & 120& 1,013,660.86= EEV$_{19-20}$\\\bottomrule      
    \end{tabular}
    \caption{Deterministic and Two-Stage Stochastic Results - Regression Tree and Specific LOS - Year Specific}
    \label{tab:Results5}
\end{table}

















\subsection{Classification Tree and Average LOS}
The third method utilises the classification tree displayed by Figure \ref{fig:finalclasstree}. The classification tree yielded 30 patient groupings, with the majority of patients falling into one of two categories. To generate the demands recall Equation \eqref{eq:treedemand1}, to calculate the D$_{s,r}$ variable:

\begin{equation}
        D_{s,r} = \sum\limits_{h \in r} D_{s,h} = \frac{\text{Number of Patients}_{s,h}*\text{Node Average LOS}}{1096} \tag{\ref{eq:treedemand1} revisited}\\
\end{equation}

For patients who fell into a `$<$1' node, meant the majority of patients were discharged on the same day of admittance. However, for all cases, there were patients who were assigned this category even if they did not meet this criteria. This means the average LOS will not be zero days to account for these patients.

Consider node nine on the CART tree where a patient falls into this category if they meet the following seven criterion:
\begin{enumerate}
    \item Admission method = Elective - waiting list
    \item Specialty $\neq$ Trauma \& Orthopaedic
    \item Speciality $\neq$ General Surgery
    \item Specialty $\neq$ Urology
    \item Specialty $\neq$ Ear Nose \& Throat
    \item Specialty $\neq$ Gynaecology
    \item Specialty = Respiratory
\end{enumerate}

For this node, the majority of patients fall into the `$<$1' category, and the average LOS is less than one (0.913918 days). This value can then be used to calculate the average daily demand required for each specialty and hospital within this node (Table \ref{tab:classnodeexample}). The associated demands are presented in Table \ref{apptab:LinkedDemands6}, within the Appendix.

\begin{table}[h!]
    \centering
    \begin{tabular}{ccccc}\toprule
        \textbf{Hospital} & \textbf{Specialty} & \textbf{Count} & \textbf{Average LOS} & \textbf{Average Daily Demand}  \\\midrule
         Nevill Hall Hospital & Respiratory &   690 & 0.913918 & 0.575368 \\
Royal Gwent Hospital & Respiratory &   553 &0.913918 & 0.461128 \\ \bottomrule
    \end{tabular}
    \caption{Classification Tree Node 9 - Average LOS}
    \label{tab:classnodeexample}
\end{table}

Table \ref{tab:Results6} presents the results utilising these demands, generating an EV value of $\pounds$ 948,153.76, by deploying 1027 beds and 648 staff. Similar to other results, fewer beds are deployed compared to the averages generated by Excel and Python. These beds are also deployed to different hospital locations, which causes a large decrease in the overall objective function. The location of these beds can be seen in Figure \ref{} \hl{Reference this APP14A} in the Appendix.


\begin{table}[h!]
    \centering
    \begin{tabular}{cccccl}\toprule
 & \multicolumn{2}{l}{\textbf{Total Beds}} & \multicolumn{2}{c}{\textbf{Total Staff}} & \multirow{2}{*}{\textbf{Objective Value ($\pounds$)}}\\ \cmidrule(lr){2-3} \cmidrule(lr){4-5}
 & xbed           & ubed          & xstaff         & ustaff         \\ \midrule
    Deterministic      & 1027 & - & 648 & - & 948,153.76 = EV \\ \midrule
    Stochastic &998& 130& 621 & 105&  980,939.52 = RP \\ \midrule
    Test A & 1027 & 96 & 648 & 111 & 1,001,995.00 = EEV \\\bottomrule
    \end{tabular}
    \caption{Deterministic and Two-Stage Stochastic Results - Classification Tree and Average LOS - Three Years}
    \label{tab:Results6}
\end{table}

The VSS was calculated to be 2.14\%, showing that with classification trees generating the demand, savings can still be made using this method. The results deployed similar numbers of beds compared to the Excel and Python models, however, the objective values are considerably smaller. One reason for this could be due fewer beds being deployed in the more expensive units as the LOS's are shorte, and therefore have smaller daily demands.

Further analysing on a year to year case, Equation \eqref{eq:treedemand2} determines how each of the demands are calculated. Once summed across each node, the formulated demand can be inputted into the deterministic and two-stage stochastic models (Tables \ref{apptab:LinkedDemands7a} and \ref{apptab:LinkedDemands7b}).

\begin{equation}
        D_{s,r,year} = \sum\limits_{h \in r} D_{s,h,year} = \frac{\text{Number of Patients}_{s,h}* {\text{Node Average LOS}}_{year}}{\text{Number of Days in Year}} \tag{\ref{eq:treedemand2} revisited}
\end{equation}

Table \ref{tab:Results7} displays the EV, RP and EEV results, showing year to year the objective value changes. As with the previous experiment, the objective values produced are lower, despite similar numbers of beds being deployed. The locations of these beds can be seen within Figures \ref{} \hl{reference these figures here}. The VSS ranges from 1.64\% to 2.27\%, showing the benefit of using the two-stage stochastic model. These results show the benefit in planning on a year to year scale rather than grouping into three years. 

\begin{table}[h!]
    \centering
    \begin{tabular}{ccccccl}\toprule
 & \multirow{2}{*}{\textbf{Year}}& \multicolumn{2}{l}{\textbf{Total Beds}} & \multicolumn{2}{c}{\textbf{Total Staff}} & \multirow{2}{*}{\textbf{Objective Value ($\pounds$)}}\\ \cmidrule(lr){3-4} \cmidrule(lr){5-6}
&& xbed           & ubed          & xstaff         & ustaff         \\ \midrule
     \multirow{3}{*}{Deterministic} & 2017 - 2018 & 1008  & - & 603 & - & 916,529.36 =  EV$_{17-18}$ \\ 
      & 2018 - 2019 & 1030& - & 624  & - &  942,069.88 =  EV$_{18-19}$ \\
      & 2019 - 2020 & 1023  & - & 624 & - &   935,425.88 =  EV$_{19-20}$\\ \midrule
     \multirow{3}{*}{Stochastic} & 2017 - 2018 & 979 & 131 &588  & 93 &  948,852.10 =  RP$_{17-18}$ \\ 
      & 2018 - 2019 & 1003  &  132 & 612 & 99 &980,984.24 = RP$_{18-19}$ \\
      & 2019 - 2020 &998 &119  &609 &93 &  972,357.90 = RP$_{19-20}$\\ \midrule    
     \multirow{3}{*}{Test A} & 2017 - 2018 & 1008 & 93 & 603  &99  & 966457.10 = EEV$_{17-18}$ \\ 
      & 2018 - 2019& 1030 & 99 &  624& 114 & 997,552.42 = EEV$_{18-19}$ \\
      & 2019 - 2020 & 1023 & 93  & 624 &102 &  988,306.28 = EEV$_{19-20}$\\ \bottomrule       
    \end{tabular}
    \caption{Deterministic and Two-Stage Stochastic Results - Classification Tree and Average LOS - Year Specific LOS}
    \label{tab:Results7}
\end{table}



These results can be extended to include the average LOS per year, rather than using the node average for the model. This utilises Equation \eqref{eq:treedemand3} to calculate the demands shown in Tables \ref{apptab:LinkedDemands8a} and \ref{apptab:LinkedDemands8b}. By using this method, should ensure that specialty LOS is taken into consideration, especially for specialties which have longer LOS's.

\begin{equation}
        D_{s,r,year} = \sum\limits_{h \in r} D_{s,h,year} = \frac{\text{Number of Patients}_{s,h}*\text{Node Average LOS}_{s,h,year}}{\text{Number of Days in Year}} \tag{\ref{eq:treedemand3} revisited}
\end{equation}

The results can be seen within Table \ref{tab:Results8}, and visualised within Figures \ref{} \hl{reference} in the Appendix. The results deploy a larger quantity of beds and staff staff in the year 2017 - 2018 compared to using the average node LOS. However, from 2018, fewer beds and staff are deployed, resulting in a lower objective value.

\begin{table}[h!]
    \centering
    \begin{tabular}{ccccccl}\toprule
 & \multirow{2}{*}{\textbf{Year}}& \multicolumn{2}{l}{\textbf{Total Beds}} & \multicolumn{2}{c}{\textbf{Total Staff}} & \multirow{2}{*}{\textbf{Objective Value ($\pounds$)}}\\ \cmidrule(lr){3-4} \cmidrule(lr){5-6}
&& xbed           & ubed          & xstaff         & ustaff         \\ \midrule
     \multirow{3}{*}{Deterministic} & 2017 - 2018 & 1031  & - & 618  & - & 937,118.16 =  EV$_{17-18}$ \\ 
      & 2018 - 2019 & 1017& - & 624 & - &  933,263.88 =  EV$_{18-19}$ \\
      & 2019 - 2020 & 1006  & - & 612 & - &  922,826.44 =  EV$_{19-20}$\\ \midrule
     \multirow{3}{*}{Stochastic} & 2017 - 2018 & 1002 & 134 &591  & 99 &  969,918.66 = RP$_{17-18}$ \\ 
      & 2018 - 2019 &  993 & 117  & 609  &  93& 968,973.22 =  RP$_{18-19}$ \\
      & 2019 - 2020 & 978 & 125 & 594 & 90& 958,670.90 =  RP$_{19-20}$\\ \midrule    
     \multirow{3}{*}{Test A} & 2017 - 2018 & 1031 & 100 & 618  &105  & 991,925.70 = EEV$_{17-18}$ \\ 
      & 2018 - 2019& 1017 &93 & 614  & 111 & 986,890.08=  EEV$_{18-19}$ \\
      & 2019 - 2020 & 1006 & 91 &612  &99 & 975,063.58 = EEV$_{19-20}$\\ \bottomrule       
    \end{tabular}
    \caption{Deterministic and Two-Stage Stochastic Results - Classification Tree and Average LOS - Year Specific LOS}
    \label{tab:Results8}
\end{table}


\hl{Summarise bit here - should probably have a summary at the end of each subsection?!}


\subsection{Classification Tree and Specific LOS}


The fourth and final method also utilises the classification tree shown in Figure \ref{fig:finalclasstree}. Instead of using average LOS for each of the 30 end nodes, the LOS will be determined for each hospital and specialty within that node. Equation \eqref{eq:treedemand4} displays how each of the demands are generated.

\begin{equation}
        D_{s,r} = \sum\limits_{h \in r} D_{s,h} = \frac{\text{Number of Patients}_{s,h}*{\text{Specific LOS}_{s,h}}}{1096} \tag{\ref{eq:treedemand4} revisited}
\end{equation}

Table \ref{tab:Node2Regressiond} presents the second node within the regression tree and determines how each of the demands are generated within this node. These results in comparison with Table \ref{tab:classnodeexample}, have shown that using specific hospital and specialty LOS overall do not increase the number beds required in this node. The generated demands can be viewed in Table \ref{apptab:LinkedDemands9}. Across all nodes, the number of beds required reduce from 1,027 beds to  1,021 beds in the deterministic stage (Table \ref{tab:Results9}). 

\begin{table}[h!]
    \centering\scalebox{0.8}{
    \begin{tabular}{ccccc}\toprule 
    \textbf{Hospital} & \textbf{Specialty} & \textbf{Count}& \textbf{Average LOS} & \textbf{Average Daily Demand}\\\midrule
    Nevill Hall Hospital &  Respiratory &     690 &0.531884 & 0.3348540 \\
    Royal Gwent Hospital &  Respiratory &     553 & 1.39057& 0.7016423 \\ \bottomrule
    \end{tabular}}
    \caption{Classification Tree Node 2 - Specific LOS}
    \label{tab:Node2Regressiond}
\end{table}

Table \ref{Results9} presents the results for the deterministic and two-stage stochastic model, with the EEV also being calculated to determine the VSS.

\begin{table}[h!]
    \centering
    \begin{tabular}{cccccl}\toprule
 & \multicolumn{2}{l}{\textbf{Total Beds}} & \multicolumn{2}{c}{\textbf{Total Staff}} & \multirow{2}{*}{\textbf{Objective Value ($\pounds$)}}\\ \cmidrule(lr){2-3} \cmidrule(lr){4-5}
 & xbed           & ubed          & xstaff         & ustaff         \\ \midrule
    Deterministic      & 1021 & - & 588 & - & 1,000,653.56 = EV \\ \midrule
    Stochastic &993& 129& 573 & 90& 1,032,779.62= RP \\ \midrule
    Test A & 1021 & 97 & 588 & 105 & 1,055,076.86 = EEV \\\bottomrule
    \end{tabular}
    \caption{Deterministic and Two-Stage Stochastic Results - Classification Tree and Specific LOS - Three Years}
    \label{tab:Results9}
\end{table}

Comparing to the regression tree with average LOS results, the deterministic and two-stage stochastic objective value's were higher in the specific LOS model. This suggests that rather than using node averages, might not produce sufficient capacity for beds and staff. The VSS produced a saving of $\pounds$22,297.24 per day (2.16\%). 

These results can be analysed on a year to year basis with the specific LOS produced from each regression tree node (Tables \ref{apptab:LinkedDemands10a} and \ref{apptab:LinkedDemands10b}). To calculate the demands for each specialty and region for each year Equation \eqref{eq:treedemand5} can be applied.


\begin{equation}
        D_{s,r,year} = \sum\limits_{h \in r} D_{s,h,year} = \frac{\text{Number of Patients}_{s,h}*{\text{Specific LOS}_{s,h,year}}}{{\text{Number of Days in Year}}} \tag{\ref{eq:treedemand5} revisited}
\end{equation}

Table \ref{tab:Results5} presents the results tool has optimised the bed and staffing numbers based on the demand values. Across the years, the number of beds deployed reduces whilst the number of nurses remain constant until the final iteration. Although more beds are deployed in the first year, the second year has a larger objective value due to different specialty beds being deployed., which increases the overall cost by 0.07\%.

\begin{table}[h!]
    \centering
    \begin{tabular}{ccccccl}\toprule
 & \multirow{2}{*}{\textbf{Year}}& \multicolumn{2}{l}{\textbf{Total Beds}} & \multicolumn{2}{c}{\textbf{Total Staff}} & \multirow{2}{*}{\textbf{Objective Value ($\pounds$)}}\\ \cmidrule(lr){3-4} \cmidrule(lr){5-6}
&& xbed           & ubed          & xstaff         & ustaff         \\ \midrule
     \multirow{3}{*}{Deterministic} & 2017 - 2018 & 1019 & - & 561 & - & 980,327.32 = EV$_{17-18}$ \\ 
      & 2018 - 2019 &1010 & - &561 & - & 981,023.32 = EV$_{18-19}$ \\
      & 2019 - 2020 & 997 & - & 552 & - &  973,479.24 = EV$_{19-20}$\\\midrule
     \multirow{3}{*}{Stochastic} & 2017 - 2018 &1002 & 109 & 543 & 90 & 1,019,992.12 = RP$_{17-18}$ \\ 
      & 2018 - 2019 &977 & 129 &540 & 96& 1,011,289.90 = RP$_{18-19}$ \\
      & 2019 - 2020 & 970 & 130 & 573 &99 & 1,012,269.58 = RP$_{19-20}$\\ \midrule
      \multirow{3}{*}{Test A}& 2017 - 2018 &1019& 105&561&105&1,008,706.66 = EEV$_{17-18}$\\
      & 2018 - 2019 &1002& 92 & 543 & 105& 1,034,227.54 = EEV$_{18-19}$\\
      & 2019 - 2020 & 997 & 99 & 552 &120 &  1,033,932.78 = EEV$_{19-20}$\\\bottomrule      
    \end{tabular}
    \caption{Deterministic and Two-Stage Stochastic Results - Classification Tree and Specific LOS - Year Specific}
    \label{tab:Results5}
\end{table}



\hl{TODO:}
\begin{itemize}
    \item \hl{Add in Figures}
    \item \hl{Reference figures within text}
  %  \item \hl{Talk about classification trees and zero bed mean issue caused on average LOS? - double check this tomorrow?}
    \item \hl{Summary Sections}
    \item comparison with current?
   % \item \hl{Talk about specific results}
\end{itemize}


\newpage



\section{Scenario Analysis}
This section aims to utilise the methods of linking the prescriptive and prescriptive models and 


This sections looks at analysing specific scenarios in ABUHB as identified through collaboration with partners within the healthboard. The healthboard has three main concerns regarding changes within the future to the healthboard and how this may impact bed and staffing requirements, if no changes are made.

\hl{See if i can find data on how many beds are currently open in each ward - and see when demand is not met. - I have this in an Excel File somewhere}

\begin{enumerate}
    \item Addition of new hospital
    \item What if demand cannot be met - interactions between other areas 
    \item long term forecasting predictions
\end{enumerate}


\subsection{Addition of a new hospital}
In November 2020, a new specialist critical care hospital, known as the Grange University Hospital (GUH), opened within the healthboard with the aim of treating the most seriously ill patients and is the designated trauma centre for the area \cite{NHSWalesa}. The hospital opened with 560 beds and features a 24 hours acute assessment unit, A\&E unit and provides 24/7 emergency care for patients that need specialist and critical care. GUH opened ahead of schedule to help the relieve the pressures caused by the second wave of Covid-19 and winter seasonal pressures. The hospital is designed to treat patients who cannot be safely managed at one of the local general hospitals, and as such required specialties to be relocated throughout the healthboard. As of 2022, GUH caters for 17 specialties and as as result the number of specialties offered by other hospitals reduced. In total, the number of specialties locations offered by the hospitals reduced from 98 to 68. An updated specialty and hospital locations can be seen within Figure \ref{fig:relocation}. One main change is caused by the `Other' category, with five of the services being relocated within one of the healthboards main hospitals.


\hl{Include fig here of updated hospitals and specialties list}
\begin{figure}
    \centering
    \includegraphics{}
    \caption{Hospitals and Specialties in ABUHB, with 1 Indicating a Specialty is Present in a Given Hospital. For Cells with a \hl{X} Background Displays Where Specialties Have Closed.}
    \label{fig:relocation}
\end{figure}

In order to incorporate GUH into the model with the existing data, the assumption has been made that patients will be admitted to any hospital within the healthboard and the regional restrictions are lifted. This results in the following constraints, where the $D_{s,r}$ parameter in changed to $D_{s}$:\\
\underline{Deterministic Model}
\begin{equation}
    \sum_{h\in\mathcal{H}} x^\textnormal{bed}_{s,h}\geq D_{s}  \hspace{0.5cm} \forall  s \in \mathcal{S}
\end{equation}
\underline{Two-Stage Stochastic Model}
\begin{equation}
    \sum_{h\in\mathcal{H}} x^\textnormal{bed}_{s,h} +  \sum_{h\in\mathcal{H}} u^\textnormal{bed}_{s,h,k} \geq D_{s,k} \hspace{0.5cm} \forall s \in \mathcal{S}, k \in \mathcal{K}\\
\end{equation}

To encourage patients to be admitted to the hospital within a region they reside, a preference matrix can be introduced in which the hospital is penalised if patients attend a hospital not in the

As the regression tree nodes resulted in similar results to the OpenSolver averages, this method will be used to generate the demands moving forward. 

The specific nodes can have values increased or decreased as to what potential future scenarios are likely to involve, rather than simply increasing or decreasing the overall demand. 




\subsection{Adding in Grange}
% The final scenario investigates the effect on current services if a new hospital is added into the region. Within ABUHB a specialist critical care centre known as The Grange University Hospital, was opened in 2022, (\hl{check}), with the aim to centralise existing services within the healthboard. Therefore

\hl{think I can merge these two together?}
\subsection{Flexibility in moving patients to different regions}
\begin{itemize}
    \item How would services work if there was an overlap, with a Preference matrix to allow movement between services - maybe? 
\end{itemize}
\subsection{M- Penalty}
\begin{itemize}
    \item Soft constraints that if demand is not met, then do not need to open ward but there is an additional cost
\end{itemize}
\subsection{Long term predictions}
\subsubsection{Re-evaluating the current setup}




\section{Generalizability of results}
\begin{itemize}
\item Excel tool has been provided - for healthboard in Chap
\item Python Tool has been provided to enable different specailties  hospitals etc.
    \item Model is adaptable
    \item used for any healthboard region size
    \item applied to other age groupings - esp with cart as this can determine different los groupings
    \item 
\end{itemize}


\end{document}
